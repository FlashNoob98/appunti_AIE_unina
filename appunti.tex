%\documentclass[a4paper,11pt]{article}
\documentclass[a4paper,11pt]{scrartcl}

\usepackage[utf8]{inputenc}
\usepackage[italian]{babel}

\usepackage{graphicx} %for includng images
\graphicspath{{img/}}

\usepackage{siunitx} % package for deciBel and other units

\usepackage{amsmath} % package for ``cases'' and other matemagical stuff

\usepackage[american]{circuitikz} % circuit drawer
\ctikzset{/tikz/circuitikz/bipoles/length=1cm} %dimensione componenti

\usepackage{subcaption}  %per immagini multiple

%\usepackage{xcolor} % Colori!
\usepackage{colortbl} %colori nelle tabelle!!

\usepackage{multirow} %righe doppie nelle tabelle

\usepackage{makecell} %multirow box

\usepackage{hyperref}

\usepackage{cancel} %utile a cancellare termini nelle equazioni
\hypersetup{ % vedi https://it.overleaf.com/learn/latex/hyperlinks
    colorlinks=true,
    linkcolor=blue,
    filecolor=magenta,      
    urlcolor=blue
}
\title{Appunti di Apparecchi ed Impianti Elettrici}
\author{Daniele Olivieri}
\date{}

\pdfinfo{%
  /Title    ()
  /Author   ()
  /Creator  ()
  /Producer ()
  /Subject  ()
  /Keywords ()
}
%\includeonly{06_19_03_esercizi}
\begin{document}
\maketitle
Ti piacciono i miei appunti? Saranno sempre liberi e gratuiti ma puoi sostenermi con una donazione cliccando \href{https://www.paypal.com/donate?hosted_button_id=7KELP768NJSYW}{QUI}

Puoi accedere ai codici sorgente seguendo questo \href{https://github.com/FlashNoob98/appunti_principi_II_unina}{link} invece.
\setcounter{tocdepth}{2}
\tableofcontents
\setlength\arrayrulewidth{1.2pt} %larghezza righe tabelle
\include{01_generalità_sistema_elettrico}

Work in progress...
\end{document}
