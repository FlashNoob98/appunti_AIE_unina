\section{Generalità sul Sistema Elettrico per l'Energia}
Fino al 16 marzo 1999 il Sistema Elettrico per l'Energia (abbreviato SEE) era esercito in modalità
monopolistica dall'ENEL, successivamente al Decreto Bersani (D.L. n\textdegree\ 79) invece
si era provveduto alla \textit{liberalizzazione del mercato dell'energia}.

L'unico monopolio rimasto al momento è quello affidato alla Trasmissione dell'energia, detto
\textit{monopolio naturale} e affidato a TERNA.

\paragraph{Impianti di produzione}
Vengono suddivisi in impianti di produzione \textit{Concentrata} e \textit{Distribuita}.
Esistono varie tipologie di impianti:
\begin{itemize}
\item Impianti Termoelettrici
\subitem -- Con turbine a \textit{vapore} (con vari tipi di combustibili tradizionali e nucleari)
\subitem -- Con turbine a \textit{gas}
\subitem -- A ciclo combinato \textit{gas-vapore}
\subitem -- Geotermoelettrici
\item Impianti Idroelettrici
\subitem -- Di tipo tradizionale
\subitem -- Di produzione e pompaggio
\end{itemize}

Gli impianti \textit{Distribuiti} funzionano solitamente con energia
\begin{itemize}
\item Solare
\item Eolica
\item Idraulica
\end{itemize}

\paragraph{Impianti di Trasmissione}
Gli impianti di trasmissione vengono definiti monopolio naturale, la sua duplicazione non sarebbe 
conveniente a causa degli elevati costi di realizzazione e gestione.
Gli impianti di trasmissione sono solitamente eserciti a \SI{380}{\kilo\volt} e \SI{220}{\kilo\volt}.
\begin{figure}[h]
\centering
\begin{circuitikz}
\draw (0,0) node[ground]{} to [sV,l=$G_1$] (1,0) 
                           to [multiwire] (1.5,0)
                           to [oosourcetrans] (2.5,0)
                           to [multiwire] (3,0) node[anchor=south]{1} to (6,2)  node[vcc]{2}
                           to (9,0) node[anchor=south]{3} to[multiwire] (9.5,0)
                           to [oosourcetrans] (10.5,0)
                           to [multiwire] (11,0)
                           to [sV, l=$G_2$] (12,0) node[ground]{};
\draw (3,0) to (6,-2) node[vee,anchor=north east]{} to (6,-2) node[vcc,anchor=north west]{} 
(6,-2.4)node[anchor = north]{4}
         (6,-2) to (9,0) to(3,0)
;
\draw (6,-2) to (6,2);
\end{circuitikz}
\caption{Schema tipico di impianto di trasmissione}
\end{figure}

I generatori $G_1$ e $G_2$ sono generatori concentrati, i nodi 2 sono di utilizzazione, ossia connettono la rete di distribuzione a più bassa tensione.
Il nodo 4 rappresenta invece il collegamento con un'altra rete, ad esempio estera.

\paragraph{Impianti di Distribuzione}
Si parla di distribuzione \textit{primaria} se esercita a tensioni comprese tra \SI{150}{\kilo\volt} o 
\SI{132}{\kilo\volt}, alimenta grandi clienti industriali oppure connette impianti di distribuzione
\textit{secondaria} abbassando la tensione a \SI{20}{\kilo\volt} (Media Tensione MT).
Infine un'ulteriore parte della distribuzione secondaria viene chiamata distribuzione in 
Bassa Tensione (BT) e raggiunge le piccole utenze con una tensione di \SI{400}{\volt}.

La rete di distribuzione inizia quindi con una tensione di \SI{380}{\kilo\volt} per poi
raggiungere una \textit{stazione primaria} o officina, che alimenta la distribuzione primaria 
abbassando la tensione a 150 o \SI{130}{\kilo\volt}; a valle della rete di distribuzione primaria
si trovano i trasformatori AT/MT che hanno lo scopo di ridurre ulteriormente la tensione a
\SI{20}{\kilo\volt}.
Infine una cabina MT/BT abbassa la tensione al valore esercito ai singoli consumatori, ossia 
\SI{400}{\volt} e la distribuzione in bassa tensione raggiunge i clienti.

\paragraph{Sistema di utilizzazione}
Costituito da cavi elettrici che si collegano alle stazioni primarie, alle stazioni AT/MT o 
alle cabine MT/BT

I sistemi elettrici industriali e i sistemi di trasporto sono spesso alimentati
da stazioni a media o alta tensione. Conterranno molto probabilmente sezioni a tensione
diversa con relative linee e cabine di trasformazione.

Gli impianti industriali sono composti generalmente da più utenti e possono essere alimentati
da complesse reti ad anello e avere come utilizzatori grandi motori trifase
in media tensione o un numero molteplice di motori di taglia più piccola.

Con il Decreto Legge 79/99 si è passati da un sistema elettrico \textit{verticale} dal produttore
all'utilizzatore ad un sistema \textit{orizzontale}, non esiste più un verso 'unidirezionale' della 
potenza dal generatore all'utilizzatore ma si deve ottimizzare la rete e i sistemi di protezione
affinchè funzionino in entrambi i versi, anche gli utenti diventano produttori di energia.
È inoltre importante saper gestire l'intermittenza dell'energia elettrica prodotta dalle fonti 
rinnovabili mediante opportuni sistemi di accumulo.

